\documentclass[11pt,a4paper,roman, colorlinks,linkcolor=true]{moderncv}        
\moderncvstyle{classic}                             % style options are 'casual' (default), 'classic', 'banking', 'oldstyle' and 'fancy'
\setlength{\hintscolumnwidth}{2.3cm} % if you want to change the width of the column with the dates
\moderncvcolor{black}                               % color options 'black', 'blue' (default), 'burgundy', 'green', 'grey', 'orange', 'purple' and 'red'

\usepackage[osf,sc]{mathpazo}
\linespread{1.05}  
\usepackage{inconsolata}
%\nopagenumbers{}                                  % uncomment to suppress automatic page numbering for CVs longer than one page
\usepackage[utf8]{inputenc}                       % if you are not using xelatex ou lualatex, replace by the encoding you are using
\usepackage[scale=0.88]{geometry}
%\setlength{\makecvtitlenamewidth}{10cm}           % for the 'classic' style, if you want to force the width allocated to your name and avoid line breaks. be careful though, the length is normally calculated to avoid any overlap with your personal info; use this at your own typographical risks...
\usepackage{courier}
\usepackage{xpatch}% http://ctan.org/pkg/xpatch

% personal data
\name{Fran}{Bartoli\'{c}}
\title{\textcolor{gray}{ \normalsize PhD candidate in astrophysics 
working on\newline probabilistic modelling of time-series data.\newline Interested 
in applying my skills to solving\newline exciting new problems in an industry setting.}}
\extrainfo{
    \normalfont \textbf{website:} \href{https://fbartolic.github.io}{fbartolic.github.io}\\
 \normalfont \textbf{email:} fb90 \emph{at} ast-andrews.ac.uk\\
 \normalfont \textbf{github:} \href{https://github.com/fbartolic}{fbartolic}\\
 \normalfont \textbf{linkedin:} \href{https://www.linkedin.com/in/fbartolic/}{fbartolic}\\
 \normalfont \textbf{location:} Oxford, UK}

\begin{document}

% FONT SIZES
\renewcommand*{\titlefont}{\fontsize{10}{14}\mdseries\upshape}
\renewcommand*{\namefont}{\fontsize{32}{34}\mdseries\upshape}
\xpatchcmd{\cvitem}{#3}{\small #3}{}{}
\patchcmd{\section}{0.95ex}{0.5ex}{}{}
\makecvtitle

\section{Personal information}
\cvitem{Nationality}{Croatian}
\cvitem{Languages}{English (Fluent), Croatian (Native)}

\definecolor{links}{HTML}{0088cc}
\hypersetup{urlcolor=links}
\section{Experience}
\cventry{02/2020--06/2020}{Research analyst}{Center for Computational Astrophysics (CCA), Flatiron Institute (Simons Foundation)}
{New York, USA}{}{
    \begin{itemize}
        \item Developed a probabilistic \href{https://speakerdeck.com/fbartolic/inferring-a-time-dependent-map-of-io-from-occultations-and-phase-curves}{model} for inferring time-variable two dimensional surface maps of exoplanets given one-dimensional time series data, using latent Nonnegative Matrix Factorization (NMF) and variational inference.
        \item Wrote code in \textsf{Python} using the \textsf{PyMC3}, \textsf{Numpyro} and \textsf{JAX} libraries.
        \item Worked in a team with two other scientists in the \href{https://www.simonsfoundation.org/flatiron/center-for-computational-astrophysics/astronomical-data/}{Astronomical Data group}.
\end{itemize}}

\section{Skills}
\cvitem{Programming}{\textsf{Python}, \textsf{C/C++}.}
\cvitem{Tools}{\textsf{PyMC3}, \textsf{Pyro}, \textsf{JAX}, \textsf{theano}, \textsf{Stan}, \textsf{scikit-learn}, \textsf{PyTorch}, \textsf{Jupyter}, \textsf{Pandas}, \textsf{matplotlib}, SQL, R,
\textsf{Git}, continuous integration, \textsf{Vim}, \textsf{Linux}, HTML \& CSS.}
\cvitem{Statistics \& ML}{Probabilistic modelling, general linear models, MCMC, variational inference, Bayesian model comparison, Gaussian processes, Frequentist statistics, neural networks, normalizing flows.}
\cvitem{Other}{I have given \href{https://speakerdeck.com/fbartolic}{talks} at international conferences and meetings and worked on projects in a team. I have tutored undergraduates in astronomy and have given talks to members of the public. 
I am good at describing complex statistical methods to non-experts.}

\section{Education}
\cventry{2017--2022 (expected)}{Ph.D. Astrophysics}{University of St Andrews}
{St Andrews, Scotland}{}{    \begin{itemize}
            \item Conducted research into Bayesian approaches to modeling astrophysical time-series data for the purpose of detecting stars, exoplanets and black holes in a regime where priors and expert knowledge are crucial.
            \item Wrote open-source Python package \href{https://caustic.readthedocs.io/en/latest/}{caustic} for fitting gravitational microlensing events using Hamiltonian Monte Carlo and Nested Sampling.
       \item Given talks at international conferences and workshops.
           \item Took courses related in machine learning and data science.
\end{itemize}}
\cventry{2015--2017}{M.Sc. Physics with Astrophysics}{University of Rijeka}
{Rijeka, Croatia}{}{\begin{itemize}\item Cumulative GPA: 4.7/5. 
            \item Took courses in theoretical physics and astrophysics.
\item Worked on a theoretical research \href{https://github.com/fbartolic/master_thesis}{project} in astrophysics for 7 months at Lund University in Sweden.\end{itemize}}
\cventry{2012--2015}{B.Sc. Physics}{University of Split}{Split, Croatia}{}
{\begin{itemize}\item Cumulative GPA: 4.5/5.
            \item Took courses in theoretical physics and computer science.
            \item Exchange semester at Lund University in Sweden.
\end{itemize}}

\section{Publications}
\cvitem{2021}{\textbf{F. Bartoli\'{c}} \& M. Dominik (in prep). Statistical challenges in modelling
gravitational microlensing events.}
\cvitem{2021}{\textbf{F. Bartoli\'{c}}, R. Luger, D. Foreman-Mackey (in prep). Occultation mapping of Io\textquotesingle s surface in the near-infrared II: Inferring dynamic maps}
\cvitem{2021}{R. Luger, E. Agol, \textbf{F. Bartoli\'{c}}, D. Foreman-Mackey. Analytic Light Curves in Reflected Light: Phase Curves, Occultations, and Non-Lambertian Scattering for Spherical Planets and Moons, \href{https://arxiv.org/abs/2103.06275}{arXiv:2103.06275}.}
\cvitem{2021}{\textbf{F. Bartoli\'{c}}, R. Luger, D. Foreman-Mackey . Occultation mapping of Io\textquotesingle s surface in the near-infrared I: Inferring static maps, \href{https://arxiv.org/abs/2103.03758}{arXiv:2103.03758}.}
\cvitem{2020}{N. Golovich, W. Dawson, \textbf{F. Bartoli\'{c}}, et al. A Reanalysis of Public OGLE-III and IV Gravitational Microlensing Events, \href{https://arxiv.org/abs/2009.07927}{arXiv:2009.07927}.}
\cvitem{2018}{V. Bozza, E. Bachelet, \textbf{F. Bartoli\'{c}}, T. M. Heintz, 
A. R. Hoag, and M. Hun-dertmark.  \emph{VBBINARYLENSING: a public package 
for microlensing light-curve  computation.}, 2018, MNRAS, 479, 5157. doi:10.1093/mnras/sty1791}

\section{Awards, Competitions and Honors}
\cvitem{2019}{\emph{Arthur Maitland Prize} for the best talk, University of St Andrews.}
\cvitem{2015}{\emph{Dean's Award for undergraduate academic excellence}, 
University of Split.}

\section{Hobbies \& Interests}
\cvitem{}{Cooking, reading, complexity science, localism, Effective Altruism.}
\end{document}
