\documentclass[11pt,a4paper,roman, colorlinks,linkcolor=true]{moderncv}        
\moderncvstyle{classic}                             % style options are 'casual' (default), 'classic', 'banking', 'oldstyle' and 'fancy'
\setlength{\hintscolumnwidth}{3.5cm} % if you want to change the width of the column with the dates
\moderncvcolor{black}                               % color options 'black', 'blue' (default), 'burgundy', 'green', 'grey', 'orange', 'purple' and 'red'

\usepackage[osf,sc]{mathpazo}
\linespread{1.05}  
\usepackage{inconsolata}
%\nopagenumbers{}                                  % uncomment to suppress automatic page numbering for CVs longer than one page
\usepackage[utf8]{inputenc}                       % if you are not using xelatex ou lualatex, replace by the encoding you are using
\usepackage[scale=0.8]{geometry}
%\setlength{\hintscolumnwidth}{3cm}                % if you want to change the width of the column with the dates
%\setlength{\makecvtitlenamewidth}{10cm}           % for the 'classic' style, if you want to force the width allocated to your name and avoid line breaks. be careful though, the length is normally calculated to avoid any overlap with your personal info; use this at your own typographical risks...
\usepackage{courier}
%\usepackage{fontspec}
%\usepackage{fontawesome}

% personal data
\name{Fran}{Bartoli\'{c}}
\title{\textcolor{gray}{ \normalsize PhD candidate in astrophysics 
working on\newline probabilistic modelling of time-series data.\newline Interested 
in applying my skills to solving\newline exciting new problems in an industry setting.}}
\extrainfo{
    \normalfont \textbf{website:} \href{https://fbartolic.github.io}{fbartolic.github.io}\\
 \normalfont \textbf{email:} fb90 \emph{at} ast-andrews.ac.uk\\
 \normalfont \textbf{github:} \href{https://github.com/fbartolic}{fbartolic}\\
 \normalfont \textbf{location:} Oxford, UK}
%\extrainfo{%
%\faTwitter \, @fbartolic\\ 
%\faGithub \, fbartolic}

%\AfterPreamble{\hypersetup{
%  urlcolor=red,
%}}

\begin{document}
\renewcommand*{\titlefont}{\fontsize{14}{18}\mdseries\upshape}
\makecvtitle
\section{Personal information}
\cvitem{Nationality}{Croatian}
\cvitem{Languages}{English (Fluent), Croatian (Native)}

\definecolor{links}{HTML}{0088cc}
\hypersetup{urlcolor=links}
\section{Experience}
\cventry{09/2017--today}{Ph.D. project}{School of Physics \& Astronomy, University of St Andrews}
{Scotland}{}{In my ongoing PhD work I build probabilistic models of astrophysical time series data 
using advanced Bayesian statistical methods for the purpose of inferring properties of exoplanets.
To implement the models I write open-source \href{https://caustic.readthedocs.io/en/latest/}{code} in \textsf{Python}.
I am particularly interested in building interpretable models in a regime where prior information and expert knowledge cannot be neglected.
The methods I use in my research include Hamiltonian Monte Carlo, Gaussian Processes, linear models, Variational Inference and automatic differentiation.}

\cventry{02/2020--06/2020}{Research analyst}{
    Center for Computational Astrophysics (CCA), Flatiron Institute (Simons Foundation)}
{New York, USA}{}{I spent 5 months working in a team with two other scientists on a probabilistic model for inferring two-dimensional surface maps of exoplanets given only one-dimensional time series measurements.
This \href{https://github.com/fbartolic/volcano/blob/master-pdf/paper/paper.pdf}{project} incorporated cutting edge methods on the intersection of statistics, machine learning and computational astrophysics.}

\section{Skills}
\cvitem{Programming}{\textsf{Python}, \textsf{C/C++}.}
\cvitem{Tools and libraries}{\textsf{PyMC3}, \textsf{Pyro}, \textsf{JAX}, \textsf{theano}, \textsf{Stan}, \textsf{scikit-learn}, \textsf{PyTorch}, \textsf{Jupyter}, \textsf{Pandas}, \textsf{matplotlib}.}
\cvitem{Statistics \& ML}{Probabilistic modelling, general linear models, MCMC, variational inference, Bayesian decision making, Gaussian processes, frequentist statistics, neural networks.}
\cvitem{Other technical}{\textsf{Git}, continuous integration, \textsf{Vim}, \textsf{Linux}, HTML \& CSS.}
\cvitem{Communication}{I have given \href{https://speakerdeck.com/fbartolic}{talks} at international conferences and meetings, along with working on projects at different scales. As part of my PhD I have tutored undergraduates in astronomy and have given talks to members of the public, having to describe complex statistical methods to non-experts.}

\section{Education}
\cventry{2017--2022 (expected)}{Ph.D. Astrophysics}{University of St Andrews}
{St Andrews, Scotland}{}{This position is a part of a \href{https://www.scotdist.ac.uk/industry/student-placements}{scheme} funded by the UK government whose goal is to produce PhD graduates with data science skills relevant to industry positions.}
\cventry{2015--2017}{M.Sc. Physics with Astrophysics}{University of Rijeka}
{Rijeka, Croatia}{}{Cumulative GPA: 4.7/5. Courses in theoretical physics. Exchange \href{https://github.com/fbartolic/master_thesis}{project} at Lund University in Sweden where I worked on an independent research project for 7 months.}
\cventry{2012--2015}{B.Sc. Physics}{University of Split}{Split, Croatia}{}
{Cumulative GPA: 4.5/5. Courses in theoretical physics and programming. Exchange semester in Lund, Sweden.}


\section{Publications}
\cvitem{2021}{\textbf{F. Bartoli\'{c}} \& M. Dominik (in prep). Statistical challenges in modelling
gravitational microlensing events.}
\cvitem{2021}{\textbf{F. Bartoli\'{c}}, R. Luger, D. Foreman-Mackey (in prep). Occultation mapping of Io\textquotesingle s surface in the near-infrared II: Inferring dynamic maps}
\cvitem{2021}{\textbf{F. Bartoli\'{c}}, R. Luger, D. Foreman-Mackey (in prep). Occultation mapping of Io\textquotesingle s surface in the near-infrared I: Inferring static maps}
\cvitem{2021}{R. Luger, E. Agol, \textbf{F. Bartoli\'{c}}, D. Foreman-Mackey (in prep). Analytic Light Curves in Reflected Light: Phase Curves, Occultations, and Non-Lambertian Scattering for
Spherical Planets and Moons.}
\cvitem{2020}{N. Golovich, W. Dawson, \textbf{F. Bartoli\'{c}}, et al. A Reanalysis of Public OGLE-III and IV Gravitational Microlensing Events, arXiv:2009.07927}
\cvitem{2018}{V. Bozza, E. Bachelet, \textbf{F. Bartoli\'{c}}, T. M. Heintz, 
A. R. Hoag, and M. Hun-dertmark.  \emph{VBBINARYLENSING: a public package 
for microlensing light-curve  computation.}, 2018, MNRAS, 479, 5157. doi:10.1093/mnras/sty1791}

\section{Awards, Competitions and Honors}
\cvitem{2019}{\emph{Arthur Maitland Prize} for the best talk, University of St Andrews.}
\cvitem{2015}{\emph{Dean's Award for undergraduate academic excellence}, 
University of Split.}

\section{Hobbies \& Interests}
\cvitem{}{Cooking, reading, complexity science, localism, Effective Altruism.}
\end{document}
