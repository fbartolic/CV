\documentclass[11pt,a4paper,roman]{moderncv}        
\moderncvstyle{classic}                             % style options are 'casual' (default), 'classic', 'banking', 'oldstyle' and 'fancy'
\moderncvcolor{black}                               % color options 'black', 'blue' (default), 'burgundy', 'green', 'grey', 'orange', 'purple' and 'red'
\usepackage[osf,sc]{mathpazo}
\linespread{1.05}  
\usepackage{inconsolata}
%\nopagenumbers{}                                  % uncomment to suppress automatic page numbering for CVs longer than one page
\usepackage[utf8]{inputenc}                       % if you are not using xelatex ou lualatex, replace by the encoding you are using
\usepackage[scale=0.8]{geometry}
%\setlength{\hintscolumnwidth}{3cm}                % if you want to change the width of the column with the dates
%\setlength{\makecvtitlenamewidth}{10cm}           % for the 'classic' style, if you want to force the width allocated to your name and avoid line breaks. be careful though, the length is normally calculated to avoid any overlap with your personal info; use this at your own typographical risks...
\usepackage{courier}
\usepackage{fontawesome}

% personal data
\name{Fran}{Bartoli\'{c}}
\title{\textcolor{gray}{ Curriculum Vitae}}
%\address{Ivana Zajca 4}{51000 Rijeka}{Croatia}
\homepage{fran.space}
\phone[mobile]{+44~(74)~609~22558}                   % optional, remove / commen
\email{fb90 at st-andrews.ac.uk} 
\extrainfo{%
\faTwitter \, @fbartolic\\ 
\faGithub \, fbartolic}

\begin{document}
\makecvtitle
\section{Personal information}
\cvitem{Nationality}{Croatian}
\cvitem{Languages}{Croatian (Native), English (Fluent)}
\section{Education}
\cventry{2017--2021 (expected)}{Ph.D. Astrophysics}{University of St Andrews}
{St Andrews, Scotland}{}{I am currently doing a PhD in Astrophysics at the University
of St Andrews in Scotland. This position is a part of a new scheme with a strong 
focus on Data Science and transferable skills. As a part of this scheme, in the middle
of my PhD I will undertake a 6 month internship in industry. The focus of my research
is on modeling gravitational microlensing events with a focus on exoplanets. I am 
primarily interested in probabilistic inference, Bayesian statistics and Machine 
Learning, both in the context of astronomy and elsewhere.}
\cventry{2015--2017}{M.Sc. Physics with Astrophysics}{University of Rijeka}
{Rijeka, Croatia}{}{Cumulative GPA: 4.7/5. I spent the first three semesters 
doing courses in theoretical 
physics and astronomy. For the last semester and during the summer, I worked on 
a research project at Lund Observatory in Sweden.}

\cventry{2012--2015}{B.Sc. Physics}{University of Split}{Split, Croatia}{}
{Cumulative GPA: 4.5/5. I spent the last semester of my bachelor's degree at
Lund University, Sweden. At Lund, I spent half the time taking courses and the other
half writing a bachelor's thesis.}
\section{Research Experience}
\cventry{09/2017--today}{Ph.D. project}{
    School of Physics \& Astronomy, University of St Andrews}
{St Andrews, Scotland}{}{I am working on modeling gravitational microlensing events
using Bayesian methods. 
The main challenge in this area is dealing with complex non-linear and degenerate 
parameter spaces, even the most sophisticated statistical methods regularly 
fail in this regime. 
My goal for this work is to approach the problem bottom-up, starting with simple fully
Bayesian models and solving one challenging problem at the time, and writing usable
open-source code in the process.
\\\\
\emph{Supervisor: Dr. Martin Dominik}}

\cventry{02/2017--09/2017}{Master's thesis project}{
Lund Observatory, Sweden}
{Lund, Sweden}{30 weeks FTE work}{
The title of my master's thesis
is \emph{Planets Orbiting Evolving Binary Stars}, and the subject is exoplanet dynamics. 
The goal of the 
project was to see what happens to circumbinary planets as the binary star evolves
up to the \emph{common-envelope} phase. In particular, I looked at the
influence of mean-motion resonance passage on the stability of orbiting planets
as the binary's orbit shrinks due to tidal interaction. I approached the
problem in two ways. First, by developing an analytical Hamiltonian model 
of high-order resonances which can be used to estimate the change in eccentricity. 
Second, by using an NBODY code \texttt{REBOUND} coupled with a 
stellar evolution code \texttt{binary\_c}
to investigate the problem in detail and validate the analytical model.
The long-term goal of this kind of work is to put dynamical constraints on circumbinary
planet evolution during late stages of stellar evolution. This is useful because
given a detection of a circumbinary planet around an evolved binary we might
be able to say if it formed with the binary and dynamically evolved to its current
orbit or if it formed during the late stages of binary evolution.\\\\
\emph{Supervisors: Dr. Alexander J. Mustill \& Prof. Dijana Dominis Prester}}

\cventry{07/2016--08/2016}{Summer student programme}{Nicolaus Copernicus
Astronomical Center}
{Warsaw, Poland}{4 weeks FTE work}{I spent a month in Warsaw working on MHD simulations of 
accretion disks with the \texttt{PLUTO} code. During my time there I ran several 
non-ideal MHD simulations of accretion disks around \emph{cataclysmic variable} 
stars in order to investigate the stability and
structure of such disks. I also wrote Python code for visualization of the Pluto 
output variables in 2D axisymmetric spherical coordinates. 
The main skills I learned there are working with Linux systems 
\& clusters, using a modern 
fluid dynamics code such as \texttt{PLUTO}, and basic 
physics of magnetohydrodynamics in the nonrelativistic regime.\\\\
\emph{Supervisors: Dr. Miljenko \v Cemelji\'c \& Prof. W\l odzimierz Klu\'zniak}
}

\cventry{01/2015--06/2015}{Bachelor's thesis project}{Lund Observatory}
{Lund, Sweden}{10 weeks FTE work}{
In my bachelor's thesis project, 
\textit{Tidal Evolution of Close-In Extra-Solar Planets}, I investigated the role
of planet induced stellar tides on the orbital evolution of planets orbiting
evolved stars. I wrote \texttt{C++} code which solves analytical models of
the tidal interactionto see how often planets spiral into
the star during late stages of stellar evolution and I compared my results 
with recent exoplanet observations. \\\\
\emph{Supervisors: Dr. Alexander J. Mustill \& Dr. Dejan Vinkovi\'{c}}
}
\section{Publications}
\cvitem{2018}{V. Bozza, E. Bachelet, \textbf{F. Bartoli\'{c}}, T. M. Heintz, 
A. R. Hoag, and M. Hun-dertmark.  \emph{VBBINARYLENSING: a public package 
for microlensing light-curve  computation.}, 479:5157–5167,  
October  2018.    doi:   10.1093/mn-ras/sty1791}

\section{Computer skills}
\cvitem{Programming}{My main language of choice is \texttt{Python}. I use it extensively together with
numerical/scientific libraries \texttt{NumPy}, \texttt{SciPy} and \texttt{matplotlib}, 
and \texttt{Jupyter} notebooks. I am also familiar with the probabilistic programming
framework \texttt{PyMC3}, which I use extensively, and machine learning 
frameworks \texttt{Tensorflow} and \texttt{PyTorch}.
Besides Python, I also use \texttt{C/C++}, and I have used
\texttt{C\#} and \texttt{MATLAB} in the past.}
\cvitem{Numerical}{I am familiar with some basic numerical methods used in physics and 
their 
implementation in \texttt{C/C++} or \texttt{Python}.}
\cvitem{Operating systems}{Linux \& Windows}
\cvitem{Other}{I use \texttt{Vim} as a text editor, \LaTeX  for writing documents, and
\texttt{VSCode} for coding.
I use \texttt{Git} version control system and \texttt{GitHub} respository
hosting service. I prefer to use open-source software in research whenever possible.}

\section{Awards, Competitions and Honors}
\cvitem{2016}{\emph{Erasmus+ Internship } scholarship}
\cvitem{2015}{\emph{Dean's Award for undergraduate academic excellence}, University of Split}
\cvitem{2014}{\emph{Erasmus+ Exchange }scholarship}

\section{Talks}
\cvitem{12/2018}{\emph{Automatic Differentiation}, talk at the weekly \emph{Code \& Cake
}
meeting, St Andrews, UK}
\cvitem{08/2018}{\emph{Bayesian modeling of gravitational microlensing events}, 
ScotDIST annual conference, Glasgow, UK}
\cvitem{05/2018}{\emph{Fitting a model to data}, talk at the weekly "Code\& Cake" 
meeting, St Andrews, UK}
\cvitem{04/2018}{\emph{Modeling gravitational microlensing events}, 
short talk at the Scottish Exoplanet/Brown Dwarf Spring meeting, St Andrews, UK}
\cvitem{09/2017}{\emph{Dynamics of circumbinary planets and the role of mean-motion 
resonances}, astronomy seminar, Lund, Sweden}
\cvitem{05/2017}{\emph{Dynamics of circumbinary planets}, Astro Journal Club 
presentation, Lund, Sweden}
\cvitem{05/2015}{\emph{The Giants of Ragnar\"ok}, bachelor's thesis public defense, 
Lund, Sweden}

\section{Organizational skills}
\cvitem{2018--today}{Co-organiser of the weekly "Code \& Cake" meeting at the School of 
Physics and Astronomy, University of St Andrews.}
\cvitem{05/2017}{LOC member for the \emph{Impacts in planetary systems} 
conference, Lund, Sweden}
\cvitem{04/2017}{Organization of \emph{Interstellar communication; a not-so-simple,
interdisciplinary response} talk for the general public by Paul Quast}
\cvitem{04/2017}{Helped with organizing the workshops at the \emph{Big Questions in Astrophysics} jubilee meeting in Lund, Sweden}
\cvitem{04/2017}{Organization of \emph{The Extremely Large Telescope} talk for 
the general public by E-ELT programme scientist Michele Cirasuolo, Lund, Sweden}

\section{Summer schools \& workshops}
\cvitem{06/2018}{\emph{Wetton Workshop}, Planning for Surprises - Data 
Driven Discovery in the era of Large Data, Christ Church Oxford, UK}
\cvitem{01/2017}{\emph{MAGIC data analysis workshop}, 3-day workshop, Rijeka, Croatia}
\cvitem{06/2016}{\emph{Extrasolar Planets: Their Formation and Evolution
, DPG Physics School}, summer school, Bad Honnef, Germany}
\cvitem{08/2015}{\emph{Summer School on Astrophysics and Astroparticles}, summer school, Petnica, Serbia }

\section{Outreach activities}
\cvitem{12/2018}{\emph{How to find exoplanets}, talk for the general public at the St 
Andrews Observatory Open Night.}
\cvitem{2017}{\emph{ALVA astronomy club}, Board member}
\cvitem{09/2014}{\emph{European Researchers' Night}, volunteer, Split, Croatia}
\cvitem{04/2014}{\emph{Festival of Science}, volunteer, Split, Croatia}

\section{Teaching experience}
\cvitem{2018}{Tutor for the course \emph{Astronomy and Astrophysics 1}, an 
introductory undergraduate astronomy course, University of St Andrews.}
\cvitem{2018}{Tutor for the course \emph{Astronomy and Astrophysics 1}, an 
introductory undergraduate astronomy course, University of St Andrews.}
\end{document}

