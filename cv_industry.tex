\documentclass[11pt,a4paper,roman]{moderncv}        
\moderncvstyle{classic}                             % style options are 'casual' (default), 'classic', 'banking', 'oldstyle' and 'fancy'
\moderncvcolor{black}                               % color options 'black', 'blue' (default), 'burgundy', 'green', 'grey', 'orange', 'purple' and 'red'
\usepackage[osf,sc]{mathpazo}
\linespread{1.05}  
\usepackage{inconsolata}
%\nopagenumbers{}                                  % uncomment to suppress automatic page numbering for CVs longer than one page
\usepackage[utf8]{inputenc}                       % if you are not using xelatex ou lualatex, replace by the encoding you are using
\usepackage[scale=0.8]{geometry}
%\setlength{\hintscolumnwidth}{3cm}                % if you want to change the width of the column with the dates
%\setlength{\makecvtitlenamewidth}{10cm}           % for the 'classic' style, if you want to force the width allocated to your name and avoid line breaks. be careful though, the length is normally calculated to avoid any overlap with your personal info; use this at your own typographical risks...
\usepackage{courier}
\usepackage{fontawesome}

% personal data
\name{Fran}{Bartoli\'{c}}
\title{\textcolor{gray}{ Curriculum Vitae}}
%\address{Ivana Zajca 4}{51000 Rijeka}{Croatia}
\homepage{fran.space}
\phone[mobile]{+44~(74)~609~22558}                   % optional, remove / commen
\email{fb90 at st-andrews.ac.uk} 
\extrainfo{%
\faTwitter \, @fbartolic\\ 
\faGithub \, fbartolic}

\begin{document}
\makecvtitle
\section{Personal information}
\cvitem{Nationality}{Croatian}
\cvitem{Languages}{Croatian (Native), English (Fluent)}

\section{Education}
\cventry{2017--2021 (expected)}{Ph.D. Astrophysics}{University of St Andrews}
{St Andrews, Scotland}{}{This PhD position is a part of a new scheme funded by the UK government
with a strong focus on producing PhD graduates with Data Science and industry relevant skills. 
In my research, I am primarily interested in building interpretable probabilistic models, 
using advanced statistical methods such as Bayesian models and Machine Learning, 
and applying them to research problems in astrophysics.}
\cventry{2015--2017}{M.Sc. Physics with Astrophysics}{University of Rijeka}
{Rijeka, Croatia}{}{Cumulative GPA: 4.7/5. I spent the first three semesters 
doing courses in theoretical 
physics and astronomy. In the final semester I worked on 
a research project at Lund University in Sweden for 7 months.}

\cventry{2012--2015}{B.Sc. Physics}{University of Split}{Split, Croatia}{}
{Cumulative GPA: 4.5/5. I took courses in theoretical physics and programming. In the final 
semester I did a short research project as an exchange student in Lund, Sweden.}

\section{Research Experience}
\cventry{09/2017--today}{Ph.D. project}{
    School of Physics \& Astronomy, University of St Andrews}
{St Andrews, Scotland}{}{I am working on modeling astrophysical time series data 
using advanced Bayesian methods.
The main challenges I am working on is how to build 
probabilistic models of complex astrophysical
phenomena in order to extract useful physical information 
about populations of extrasolar planets.
These models are complex, non-linear, and have degenerate parameter spaces, a regime where
even the most sophisticated statistical methods regularly fail. 
The statistical methods I use in my research include Gaussian Processes, Hamiltonian Monte Carlo,
linear models, and neural networks.
I am approaching the thesis project bottom-up, starting with simple interpretable models, solving 
one challenging problem at the time, and writing usable and scalable open-source code 
in the process. 
\\\\
\emph{Supervisor: Dr. Martin Dominik}}

\cventry{02/2017--09/2017}{Master's thesis project}{
Lund Observatory, Sweden}
{Lund, Sweden}{30 weeks FTE work}{
The title of my master's thesis
is \emph{Planets Orbiting Evolving Binary Stars}, and the subject of the thesis was to construct 
analytical and numerical models describing the dynamics of extrasolar planets.
In this project I developed both an analytical pen and paper model, 
and used numerical simulations to understand the relevant physical phenomenon. 
The project involved writing \texttt{Python} code for 
solving symbolic equations and making plots, and  
interfaced \texttt{Python} code with  a solver for differential equations 
written in \texttt{C++}. 
\\\\
\emph{Supervisors: Dr. Alexander J. Mustill \& Prof. Dijana Dominis Prester}}

\cventry{07/2016--08/2016}{Summer student programme}{Nicolaus Copernicus
Astronomical Center}
{Warsaw, Poland}{4 weeks FTE work}{I worked on astrophysical fluid dynamics 
simulations using a popular astrophysics code written in \texttt{C++}.
I spent my time running simulations and writing a data visualization code in \texttt{Python}.
I learned to work with Linux systems  computer \& clusters.
\\\\
\emph{Supervisors: Dr. Miljenko \v Cemelji\'c \& Prof. W\l odzimierz Klu\'zniak}
}

\section{Technical skills}
\cvitem{Programming}{My main language of choice is \texttt{Python}. I use 
it extensively together with
numerical/scientific libraries \texttt{NumPy}, \texttt{SciPy} and \texttt{matplotlib}, 
and \texttt{Jupyter} notebooks. Besides Python, I also use \texttt{C/C++}, and I have used
\texttt{C\#} and \texttt{MATLAB} in the past.}
\cvitem{Data science}{In my research I use the probabilistic programming library
\texttt{PyMC3} extensively together with \texttt{theano}. I have basic knowledge of 
\texttt{Tensorflow} and \texttt{PyTorch}.}
\cvitem{Statistics}{MCMC, Bayesian model comparison, linear models, dimensionality reduction.}
\cvitem{Other}{\texttt{Git} version control, \texttt{Vim}, \LaTeX, Linux systems, Bash shell.}

\section{Publications}
\cvitem{2019}{\textbf{F. Bartoli\'{c}} et. al. (in prep). 
Paper on building a hierarchical 
Bayesian model for predicting properties of  on going microlensing events.}
\cvitem{2019}{\textbf{F. Bartoli\'{c}} et. al. (in prep). 
Paper on Bayesian modeling of microlensing
events using Gaussian Processes and Hamiltonian Monte Carlo.}
\cvitem{2018}{V. Bozza, E. Bachelet, \textbf{F. Bartoli\'{c}}, T. M. Heintz, 
A. R. Hoag, and M. Hun-dertmark.  \emph{VBBINARYLENSING: a public package 
for microlensing light-curve  computation.}, 479:5157–5167,  
October  2018.    doi:   10.1093/mn-ras/sty1791}

\section{Awards, Competitions and Honors}
\cvitem{2016}{\emph{Erasmus+ Internship } scholarship}
\cvitem{2015}{\emph{Dean's Award for undergraduate academic excellence}, University of Split}
\cvitem{2014}{\emph{Erasmus+ Exchange }scholarship}

\section{Talks}
\cvitem{12/2018}{\emph{Automatic Differentiation}, talk at the weekly \emph{Code \& Cake
}
meeting, St Andrews, UK}
\cvitem{08/2018}{\emph{Bayesian modeling of gravitational microlensing events}, 
ScotDIST annual conference, Glasgow, UK}
\cvitem{05/2018}{\emph{Fitting a model to data}, talk at the weekly "Code\& Cake" 
meeting, St Andrews, UK}
\cvitem{04/2018}{\emph{Modeling gravitational microlensing events}, 
short talk at the Scottish Exoplanet/Brown Dwarf Spring meeting, St Andrews, UK}
\cvitem{09/2017}{\emph{Dynamics of circumbinary planets and the role of mean-motion 
resonances}, astronomy seminar, Lund, Sweden}
\cvitem{05/2017}{\emph{Dynamics of circumbinary planets}, Astro Journal Club 
presentation, Lund, Sweden}
\cvitem{05/2015}{\emph{The Giants of Ragnar\"ok}, bachelor's thesis public defense, 
Lund, Sweden}

\section{Organizational skills}
\cvitem{2018--today}{Co-organiser of the weekly "Code \& Cake" meeting at the School of 
Physics and Astronomy, University of St Andrews.}
\cvitem{05/2017}{LOC member for the \emph{Impacts in planetary systems} 
conference, Lund, Sweden}
\cvitem{04/2017}{Organization of \emph{Interstellar communication; a not-so-simple,
interdisciplinary response} talk for the general public by Paul Quast}
\cvitem{04/2017}{Helped with organizing the workshops at the \emph{Big Questions in Astrophysics} jubilee meeting in Lund, Sweden}
\cvitem{04/2017}{Organization of \emph{The Extremely Large Telescope} talk for 
the general public by E-ELT programme scientist Michele Cirasuolo, Lund, Sweden}

\section{Supervision}
\cvitem{Summer 2018}{Drew Millard, summer student at St Andrews. I supervised him on a 
project on using machine learning to classify astrophysical time series data.}

\section{Teaching experience}
\cvitem{2018}{Tutor for the course \emph{Astronomy and Astrophysics 1}, an 
introductory undergraduate astronomy course, University of St Andrews.}
\end{document}

