\documentclass[11pt,a4paper,roman]{moderncv}        
\moderncvstyle{classic}                             % style options are 'casual' (default), 'classic', 'banking', 'oldstyle' and 'fancy'
\moderncvcolor{black}                               % color options 'black', 'blue' (default), 'burgundy', 'green', 'grey', 'orange', 'purple' and 'red'
\usepackage[osf,sc]{mathpazo}
\linespread{1.05}  
\usepackage{inconsolata}
%\nopagenumbers{}                                  % uncomment to suppress automatic page numbering for CVs longer than one page
\usepackage[utf8]{inputenc}                       % if you are not using xelatex ou lualatex, replace by the encoding you are using
\usepackage[scale=0.8]{geometry}
%\setlength{\hintscolumnwidth}{3cm}                % if you want to change the width of the column with the dates
%\setlength{\makecvtitlenamewidth}{10cm}           % for the 'classic' style, if you want to force the width allocated to your name and avoid line breaks. be careful though, the length is normally calculated to avoid any overlap with your personal info; use this at your own typographical risks...
\usepackage{courier}
\usepackage{fontawesome}

% personal data
\name{Fran}{Bartoli\'{c}}
\title{\textcolor{gray}{ Curriculum Vitae}}
%\address{Ivana Zajca 4}{51000 Rijeka}{Croatia}
\homepage{fran.space}
\phone[mobile]{+44~(74)~609~22558}                   % optional, remove / commen
\email{fb90 at st-andrews.ac.uk} 
\extrainfo{%
\faTwitter \, @fbartolic\\ 
\faGithub \, fbartolic}

\begin{document}
\makecvtitle
\section{Personal information}
\cvitem{Nationality}{Croatian}
\cvitem{Languages}{Croatian (Native), English (Fluent)}

\section{Education}
\cventry{2017--2021 (expected)}{Ph.D. Astrophysics}{University of St Andrews}
{St Andrews, Scotland}{}{This PhD position is a part of a new scheme funded by the UK government
with a strong focus on producing PhD graduates with Data Science and industry relevant skills. 
In my research, I am primarily interested in building interpretable probabilistic models, 
using advanced statistical methods such as Bayesian models and Machine Learning.}
\cventry{2015--2017}{M.Sc. Physics with Astrophysics}{University of Rijeka}
{Rijeka, Croatia}{}{Cumulative GPA: 4.7/5. I spent the first three semesters 
doing courses in theoretical 
physics and astronomy. In the final semester I worked on 
a research project at Lund University in Sweden for 7 months.}

\cventry{2012--2015}{B.Sc. Physics}{University of Split}{Split, Croatia}{}
{Cumulative GPA: 4.5/5. I took courses in theoretical physics and programming. In the final 
semester I did a short research project as an exchange student in Lund, Sweden.}

\section{Research Experience}
\cventry{09/2017--today}{Ph.D. project}{
    School of Physics \& Astronomy, University of St Andrews}
{Scotland}{}{I am working on modeling astrophysical time series data 
using advanced Bayesian methods.
The main challenges I am working on is how to build 
probabilistic models of complex astrophysical
phenomena in order to extract useful physical information 
about populations of extrasolar planets.
These models are complex, non-linear, and have degenerate parameter spaces, a regime where
even the most sophisticated statistical methods regularly fail. 
The statistical methods I use in my research include Gaussian Processes, Hamiltonian Monte Carlo,
linear models, and neural networks.
I am approaching the thesis project bottom-up, starting with simple interpretable models, solving 
one challenging problem at the time, and writing usable and scalable open-source code 
in the process.} 

\cventry{02/2017--09/2017}{Master's thesis project}{
Lund University}
{Sweden}{30 weeks FTE work}{
The subject of the thesis was to construct 
analytical and numerical models describing the dynamics of extrasolar planets.
I developed both an analytical pen and paper model, 
and used numerical simulations to understand the relevant physical phenomenon. 
The project involved writing \texttt{Python} code for 
solving symbolic equations and making plots, and interfacing \texttt{Python} 
code with a solver for differential equations 
written in \texttt{C++}.}

\cventry{07/2016--08/2016}{Summer research program}{Nicolaus Copernicus
Astronomical Center}
{Warsaw, Poland}{4 weeks FTE work}{I worked on astrophysical fluid dynamics 
simulations using a popular astrophysics code written in \texttt{C++}.
I spent my time running simulations and writing a data visualization code in \texttt{Python}.
I learned to work with Linux based computer clusters.
}

\section{Skills}
\cvitem{Programming}{I have extensive experience in \texttt{Python}, together 
with libraries \texttt{NumPy}, 
\texttt{Pandas}, \texttt{matplotlib}, and \texttt{Jupyter} notebooks. 
Besides Python. I have intermediate experience with \texttt{C/C++}, and have written
code combining Python and C++. I have used \texttt{C\#} in the past.}
\cvitem{Data science}{I use the probabilistic programming library
\texttt{PyMC3} extensively together with \texttt{theano}. I have implemented simple 
machine learning models in \texttt{Tensorflow} and \texttt{PyTorch}.}
\cvitem{Statistics}{Extensive experience with Bayesian 
modeling and MCMC methods. Some experience with classification problems.}
\cvitem{Other technical}{\texttt{Git} version control, \texttt{Vim}, \LaTeX, 
\texttt{Linux} systems, \texttt{Bash} shell.}
\cvitem{Communication}{ I have given talks at conferences and meetings in academia, 
as well as talks to the general public. I have experience in giving 
tutorial to undergraduate students in astronomy. I have experience with
describing complex statistical methods to a lay audience.}
\cvitem{Organizational}{I am organizing a regular meeting on Data Science at my
department and co-organising a meeting on coding.}
\cvitem{Mentoring}{I have supervised a summer student working on a machine learning 
project in astronomy.}

\section{Publications}
\cvitem{2019}{\textbf{F. Bartoli\'{c}} et. al. (in prep). 
Building a hierarchical 
Bayesian model for astrophysical time series prediction.}
\cvitem{2019}{\textbf{F. Bartoli\'{c}} et. al. (in prep). 
Bayesian modeling of microlensing events using Gaussian Processes 
and Hamiltonian Monte Carlo.}
\cvitem{2018}{V. Bozza, E. Bachelet, \textbf{F. Bartoli\'{c}}, T. M. Heintz, 
A. R. Hoag, and M. Hun-dertmark.  \emph{VBBINARYLENSING: a public package 
for microlensing light-curve  computation.}, 479:5157–5167,  
October  2018.    doi:   10.1093/mn-ras/sty1791}

\section{Awards, Competitions and Honors}
\cvitem{2016}{\emph{Erasmus+ Internship } scholarship}
\cvitem{2015}{\emph{Dean's Award for undergraduate academic excellence}, 
University of Split}
\cvitem{2014}{\emph{Erasmus+ Exchange }scholarship}
\end{document}

