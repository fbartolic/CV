\documentclass[11pt,a4paper,roman]{moderncv}        
\moderncvstyle{classic}                             % style options are 'casual' (default), 'classic', 'banking', 'oldstyle' and 'fancy'
\moderncvcolor{black}                               % color options 'black', 'blue' (default), 'burgundy', 'green', 'grey', 'orange', 'purple' and 'red'
\usepackage[osf,sc]{mathpazo}
\linespread{1.05}  
\usepackage{inconsolata}
%\nopagenumbers{}                                  % uncomment to suppress automatic page numbering for CVs longer than one page
\usepackage[utf8]{inputenc}                       % if you are not using xelatex ou lualatex, replace by the encoding you are using
\usepackage[scale=0.8]{geometry}
%\setlength{\hintscolumnwidth}{3cm}                % if you want to change the width of the column with the dates
%\setlength{\makecvtitlenamewidth}{10cm}           % for the 'classic' style, if you want to force the width allocated to your name and avoid line breaks. be careful though, the length is normally calculated to avoid any overlap with your personal info; use this at your own typographical risks...
\usepackage{courier}
\usepackage{fontawesome}

% personal data
\name{Fran}{Bartoli\'{c}}
\title{\textcolor{gray}{ Curriculum Vitae}}
%\address{Ivana Zajca 4}{51000 Rijeka}{Croatia}
\homepage{fran.space}
\phone[mobile]{+44~(74)~609~22558}                   % optional, remove / commen
\email{fb90 at st-andrews.ac.uk} 
\extrainfo{%
\faTwitter \, @fbartolic\\ 
\faGithub \, fbartolic}

\begin{document}
\makecvtitle
\section{Personal information}
\cvitem{Nationality}{Croatian}
\cvitem{Languages}{Croatian (Native), English (Fluent)}
\section{Education}
\cventry{2017--2021 (expected)}{Ph.D. Astrophysics}{University of St Andrews}
{St Andrews, Scotland}{}{I am currently doing a PhD in Astrophysics at the University
of St Andrews in Scotland. This position is a part of a new scheme with a strong 
focus on Data Science and transferable skills. As a part of this scheme, in the middle
of my PhD I will undertake a 6 month internship in industry. The focus of my research
is on modeling gravitational microlensing events with a focus on exoplanets. I am 
primarily interested in probabilistic inference, Bayesian statistics and Machine 
Learning, both in the context of astronomy and elsewhere.}
\cventry{2015--2017}{M.Sc. Physics with Astrophysics}{University of Rijeka}
{Rijeka, Croatia}{}{Cumulative GPA: 4.7/5. I spent the first three semesters 
doing courses in theoretical 
physics and astronomy. For the last semester and during the summer, I worked on 
a research project at Lund Observatory in Sweden.}

\cventry{2012--2015}{B.Sc. Physics}{University of Split}{Split, Croatia}{}
{Cumulative GPA: 4.5/5. I spent the last semester of my bachelor's degree at
Lund University, Sweden. At Lund, I spent half the time taking courses and the other
half writing a bachelor's thesis.}

\section{Publications}
\cvitem{2018}{V. Bozza, E. Bachelet, \textbf{F. Bartoli\'{c}}, T. M. Heintz, 
A. R. Hoag, and M. Hun-dertmark.  \emph{VBBINARYLENSING: a public package 
for microlensing light-curve  computation.}, 479:5157–5167,  
October  2018.    doi:   10.1093/mn-ras/sty1791}

\section{Computer skills}
\cvitem{Programming}{My main language of choice is \texttt{Python}. I use it extensively together with
numerical/scientific libraries \texttt{NumPy}, \texttt{SciPy} and \texttt{matplotlib}, 
and \texttt{Jupyter} notebooks. I am also familiar with the probabilistic programming
framework \texttt{PyMC3}, which I use extensively, and machine learning 
frameworks \texttt{Tensorflow} and \texttt{PyTorch}.
Besides Python, I also use \texttt{C/C++}, and I have used
\texttt{C\#} and \texttt{MATLAB} in the past.}
\cvitem{Numerical}{I am familiar with some basic numerical methods used in physics and 
their 
implementation in \texttt{C/C++} or \texttt{Python}.}
\cvitem{Operating systems}{Linux \& Windows}
\cvitem{Other}{I use \texttt{Vim} as a text editor, \LaTeX  for writing documents, and
\texttt{VSCode} for coding.
I use \texttt{Git} version control system and \texttt{GitHub} respository
hosting service. I prefer to use open-source software in research whenever possible.}

\section{Awards, Competitions and Honors}
\cvitem{2016}{\emph{Erasmus+ Internship } scholarship}
\cvitem{2015}{\emph{Dean's Award for undergraduate academic excellence}, University of Split}
\cvitem{2014}{\emph{Erasmus+ Exchange }scholarship}

\section{Talks}
\cvitem{12/2018}{\emph{Automatic Differentiation}, talk at the weekly \emph{Code \& Cake
}
meeting, St Andrews, UK}
\cvitem{08/2018}{\emph{Bayesian modeling of gravitational microlensing events}, 
ScotDIST annual conference, Glasgow, UK}
\cvitem{05/2018}{\emph{Fitting a model to data}, talk at the weekly "Code\& Cake" 
meeting, St Andrews, UK}
\cvitem{04/2018}{\emph{Modeling gravitational microlensing events}, 
short talk at the Scottish Exoplanet/Brown Dwarf Spring meeting, St Andrews, UK}
\cvitem{09/2017}{\emph{Dynamics of circumbinary planets and the role of mean-motion 
resonances}, astronomy seminar, Lund, Sweden}
\cvitem{05/2017}{\emph{Dynamics of circumbinary planets}, Astro Journal Club 
presentation, Lund, Sweden}
\cvitem{05/2015}{\emph{The Giants of Ragnar\"ok}, bachelor's thesis public defense, 
Lund, Sweden}

\section{Organizational skills}
\cvitem{2018--today}{Co-organiser of the weekly "Code \& Cake" meeting at the School of 
Physics and Astronomy, University of St Andrews.}
\cvitem{05/2017}{LOC member for the \emph{Impacts in planetary systems} 
conference, Lund, Sweden}
\cvitem{04/2017}{Organization of \emph{Interstellar communication; a not-so-simple,
interdisciplinary response} talk for the general public by Paul Quast}
\cvitem{04/2017}{Helped with organizing the workshops at the \emph{Big Questions in Astrophysics} jubilee meeting in Lund, Sweden}
\cvitem{04/2017}{Organization of \emph{The Extremely Large Telescope} talk for 
the general public by E-ELT programme scientist Michele Cirasuolo, Lund, Sweden}

\section{Supervision}
\cvitem{Summer 2018}{Drew Millard, summer student at St Andrews. I supervised him on a 
project on using machine learning to classify astrophysical time series data.}

\section{Teaching experience}
\cvitem{2018}{Tutor for the course \emph{Astronomy and Astrophysics 1}, an 
introductory undergraduate astronomy course, University of St Andrews.}
\end{document}

