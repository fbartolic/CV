\documentclass[11pt,a4paper,roman, colorlinks,linkcolor=true]{moderncv}        
\moderncvstyle{classic}                             % style options are 'casual' (default), 'classic', 'banking', 'oldstyle' and 'fancy'
\setlength{\hintscolumnwidth}{3.5cm} % if you want to change the width of the column with the dates
\moderncvcolor{black}                               % color options 'black', 'blue' (default), 'burgundy', 'green', 'grey', 'orange', 'purple' and 'red'

\usepackage[osf,sc]{mathpazo}
\linespread{1.05}  
\usepackage{inconsolata}
%\nopagenumbers{}                                  % uncomment to suppress automatic page numbering for CVs longer than one page
\usepackage[utf8]{inputenc}                       % if you are not using xelatex ou lualatex, replace by the encoding you are using
\usepackage[scale=0.8]{geometry}
%\setlength{\hintscolumnwidth}{3cm}                % if you want to change the width of the column with the dates
%\setlength{\makecvtitlenamewidth}{10cm}           % for the 'classic' style, if you want to force the width allocated to your name and avoid line breaks. be careful though, the length is normally calculated to avoid any overlap with your personal info; use this at your own typographical risks...
\usepackage{courier}
%\usepackage{fontspec}
%\usepackage{fontawesome}

% personal data
\name{Fran}{Bartoli\'{c}}
\title{\textcolor{gray}{ \normalsize PhD candidate in astrophysics 
working on\newline probabilistic modeling of time-series data.\newline Interested 
in applying my skills to solving\newline problems in an industry setting.}}
\extrainfo{
    \normalfont \textbf{website:} fbartolic.github.io\\
 \normalfont \textbf{email:} fb90 \emph{at} ast-andrews.ac.uk\\
 \normalfont \textbf{github:} fbartolic\\
 \normalfont \textbf{location:} Oxford, UK}
%\extrainfo{%
%\faTwitter \, @fbartolic\\ 
%\faGithub \, fbartolic}

%\AfterPreamble{\hypersetup{
%  urlcolor=red,
%}}

\begin{document}
\renewcommand*{\titlefont}{\fontsize{14}{18}\mdseries\upshape}
\makecvtitle
\section{Personal information}
\cvitem{Nationality}{Croatian}
\cvitem{Languages}{Croatian (Native), English (Fluent)}

\definecolor{links}{HTML}{0088cc}
\hypersetup{urlcolor=links}
\section{Experience}
\cventry{09/2017--today}{Ph.D. project}{
    School of Physics \& Astronomy, University of St Andrews}
{Scotland}{}{I build probabilistic models of astrophysical time series data 
using advanced Bayesian statistical methods for the purpose of inferring properties of exoplanets.
In the process, I write open-source \href{https://caustic.readthedocs.io/en/latest/}{code} written in \textsf{Python}.
I am particularly interested in building interpretable models in a regime where prior information and expert knowledge cannot be neglected.
The methods I use in my research include Hamiltonian Monte Carlo, Gaussian Processes, linear models, Variational Inference and automatic differentiation.}

\cventry{02/2020--06/2020}{Research analyst}{
    Center for Computational Astrophysics (CCA), Flatiron Institute (Simons Foundation)}
{New York}{}{I spent 5 months at the Flatiron Institute working in a team with two other scientists on a probabilistic model for inferring the two-dimensional surfaces of exoplanets given only one-dimensional time series measurements.
This \href{https://github.com/fbartolic/volcano/blob/master-pdf/paper/paper.pdf}{project} incorporates cutting edge methods on the intersection of statistics, machine learning and computational physics.}

\section{Skills}
\cvitem{Programming}{\textsf{Python}, \textsf{C/C++}.}
\cvitem{Tools and libraries}{\textsf{PyMC3}, \textsf{Pyro}, \textsf{JAX}, \textsf{theano}, \textsf{Stan}, \textsf{scikit-learn}, \textsf{PyTorch}, \textsf{Jupyter}, \textsf{Pandas}, \textsf{matplotlib}.}
\cvitem{Statistics \& ML}{Probabilistic modeling, general linear models, MCMC, Variational Inference, Bayesian decision making, Gaussian Processes, frequentist statistics, Neural Networks.}
\cvitem{Other technical}{\textsf{Git}, \textsf{Vim}, \textsf{Linux}, basic Azure DevOps, HTML \& CSS.}
\cvitem{Communication}{I have given talks at conferences and meetings in academia
as well as talks to the general public. I have given tutorials 
for an undergraduate course in in astronomy. 
I have experience with describing complex statistical methods to a lay audience.}
\cvitem{Mentoring}{I have tutored undergraduates in astronomy.}

\section{Education}
\cventry{2017--2022 (expected)}{Ph.D. Astrophysics}{University of St Andrews}
{St Andrews, Scotland}{}{This position is a part of a \href{https://www.scotdist.ac.uk/industry/student-placements}{scheme} funded by the UK government whose goal is to produce PhD graduates with data science skills relevant to industry positions.}
\cventry{2015--2017}{M.Sc. Physics with Astrophysics}{University of Rijeka}
{Rijeka, Croatia}{}{Cumulative GPA: 4.7/5. Courses in theoretical physics. Exchange \href{https://github.com/fbartolic/master_thesis}{project} at Lund University in Sweden where I worked on an independent research project for 7 months.}
\cventry{2012--2015}{B.Sc. Physics}{University of Split}{Split, Croatia}{}
{Cumulative GPA: 4.5/5. Courses in theoretical physics and programming. Exchange semester in Lund, Sweden.}


\section{Publications}
\cvitem{2021}{\textbf{F. Bartoli\'{c}} \& M. Dominik (in prep). Statistical challenges in modeling
gravitational microlensing events.}
\cvitem{2021}{\textbf{F. Bartoli\'{c}}, R. Luger, D. Foreman-Mackey (in prep). Occultation mapping of Io\textquotesingle s surface in the near-infrared II: Inferring dynamic maps}
\cvitem{2021}{\textbf{F. Bartoli\'{c}}, R. Luger, D. Foreman-Mackey (in prep). Occultation mapping of Io\textquotesingle s surface in the near-infrared I: Inferring static maps}
\cvitem{2021}{R. Luger, E. Agol, \textbf{F. Bartoli\'{c}}, D. Foreman-Mackey (in prep). Analytic Light Curves in Reflected Light: Phase Curves, Occultations, and Non-Lambertian Scattering for
Spherical Planets and Moons.}
\cvitem{2020}{N. Golovich, W. Dawson, \textbf{F. Bartoli\'{c}}, et al. A Reanalysis of Public OGLE-III and IV Gravitational Microlensing Events, arXiv:2009.07927}
\cvitem{2018}{V. Bozza, E. Bachelet, \textbf{F. Bartoli\'{c}}, T. M. Heintz, 
A. R. Hoag, and M. Hun-dertmark.  \emph{VBBINARYLENSING: a public package 
for microlensing light-curve  computation.}, 2018, MNRAS, 479, 5157. doi:10.1093/mnras/sty1791}

\section{Awards, Competitions and Honors}
\cvitem{2019}{\emph{Arthur Maitland Prize} for the best talk, University of St Andrews.}
\cvitem{2015}{\emph{Dean's Award for undergraduate academic excellence}, 
University of Split.}
\end{document}
