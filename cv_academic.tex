\documentclass[11pt,a4paper,roman]{moderncv}        
\moderncvstyle{classic}                             % style options are 'casual' (default), 'classic', 'banking', 'oldstyle' and 'fancy'
\moderncvcolor{black}                               % color options 'black', 'blue' (default), 'burgundy', 'green', 'grey', 'orange', 'purple' and 'red'
\setlength{\hintscolumnwidth}{2.8cm} % if you want to change the width of the column with the dates
\usepackage[osf,sc]{mathpazo}
\linespread{1.05}  
\usepackage{inconsolata}
%\nopagenumbers{}                                  % uncomment to suppress automatic page numbering for CVs longer than one page
\usepackage[utf8]{inputenc}                       % if you are not using xelatex ou lualatex, replace by the encoding you are using
\usepackage[scale=0.8]{geometry}
%\setlength{\hintscolumnwidth}{3cm}                % if you want to change the width of the column with the dates
%\setlength{\makecvtitlenamewidth}{10cm}           % for the 'classic' style, if you want to force the width allocated to your name and avoid line breaks. be careful though, the length is normally calculated to avoid any overlap with your personal info; use this at your own typographical risks...
\usepackage{courier}
\usepackage{fontawesome}

% personal data
\name{Fran}{Bartoli\'{c}}
\title{\textcolor{gray}{ Curriculum Vitae}}
%\address{Ivana Zajca 4}{51000 Rijeka}{Croatia}
\homepage{fran.space}
\phone[mobile]{+44~(74)~609~22558}                   % optional, remove / commen
\email{fb90 at st-andrews.ac.uk} 
\extrainfo{%
\faTwitter \, @fbartolic\\ 
\faGithub \, fbartolic}

\begin{document}
\makecvtitle
\section{Personal information}
\cvitem{Nationality}{Croatian}
\cvitem{Languages}{Croatian (Native), English (Fluent)}
\section{Education}
\cventry{2017--2021 (expected)}{Ph.D. Astrophysics}{University of St Andrews}
{St Andrews, Scotland}{}{I am currently doing a PhD in Astrophysics at the University
of St Andrews in Scotland. This position is a part of a new scheme with a strong 
focus on Data Science and transferable skills. As a part of this scheme, during
my PhD I will undertake a 6 month internship in industry. My research
is on statistical modeling gravitational microlensing events with a focus on exoplanets. 
I am interested in probabilistic inference, Bayesian statistics and Machine 
Learning, both in the context of astronomy and elsewhere.}
\cventry{2015--2017}{M.Sc. Physics with Astrophysics}{University of Rijeka}
{Rijeka, Croatia}{}{Cumulative GPA: 4.7/5. I spent the first three semesters 
doing courses in theoretical 
physics and astronomy. In the final semester and during the summer I worked on 
a research project at Lund Observatory in Sweden.}

\cventry{2012--2015}{B.Sc. Physics}{University of Split}{Split, Croatia}{}
{Cumulative GPA: 4.5/5. I took courses in theoretical physics and programming. In the final 
semester I did a short research project as an exchange student in Lund, Sweden.}
\section{Research Experience}
\cventry{09/2017--today}{Ph.D. project}{
    School of Physics \& Astronomy, University of St Andrews}
{St Andrews, Scotland}{}{I am working on modeling gravitational microlensing events
using Bayesian statistical methods. 
The main challenge in this area is dealing with complex non-linear and degenerate 
parameter spaces, a regime where even the most sophisticated statistical methods regularly 
fail.
I am approaching the problem bottom-up, starting with simple fully
Bayesian models, solving one challenging problem at the time, and writing usable
open-source code in the process.
\\\\
\emph{Supervisor: Dr. Martin Dominik}}

\cventry{02/2017--09/2017}{Master's thesis project}{
Lund Observatory, Sweden}
{Lund, Sweden}{30 weeks FTE work}{
The title of my master's thesis
is \emph{Planets Orbiting Evolving Binary Stars}, and the subject is exoplanet dynamics. 
The goal of the 
project was to see what happens to circumbinary planets as the binary star evolves
up to the \emph{common-envelope} phase. In particular, I looked at the
influence of mean-motion resonance passage on the stability of orbiting planets
as the binary's orbit shrinks due to tidal interaction. I approached the
problem in two ways. First, by developing an analytical Hamiltonian model 
of high-order resonances which can be used to estimate the change in eccentricity. 
Second, by using an NBODY code \texttt{REBOUND} coupled with a 
stellar evolution code \texttt{binary\_c}
to investigate the problem in detail and validate the analytical model.
The long-term goal of this kind of work is to put dynamical constraints on circumbinary
planet evolution during late stages of stellar evolution. This is useful because
given a detection of a circumbinary planet around an evolved binary we might
be able to say if it formed with the binary and dynamically evolved to its current
orbit or if it formed during the late stages of binary evolution.\\\\
\emph{Supervisors: Dr. Alexander J. Mustill \& Prof. Dijana Dominis Prester}}

\cventry{07/2016--08/2016}{Summer student programme}{Nicolaus Copernicus
Astronomical Center}
{Warsaw, Poland}{4 weeks FTE work}{I spent a month in Warsaw working on MHD simulations of 
accretion disks with the \texttt{PLUTO} code. During my time there I ran several 
non-ideal MHD simulations of accretion disks around \emph{cataclysmic variable} 
stars in order to investigate the stability and
structure of such disks. I also wrote Python code for visualization of the Pluto 
output variables in 2D axisymmetric spherical coordinates. 
The main skills I learned there are working with Linux systems 
\& clusters, using a modern 
fluid dynamics code such as \texttt{PLUTO}, and basic 
physics of magnetohydrodynamics in the nonrelativistic regime.\\\\
\emph{Supervisors: Dr. Miljenko \v Cemelji\'c \& Prof. W\l odzimierz Klu\'zniak}
}

\cventry{01/2015--06/2015}{Bachelor's thesis project}{Lund Observatory}
{Lund, Sweden}{10 weeks FTE work}{
In my bachelor's thesis project, 
\textit{Tidal Evolution of Close-In Extra-Solar Planets}, I investigated the role
of planet induced stellar tides on the orbital evolution of planets orbiting
evolved stars. I wrote \texttt{C++} code which solves analytical models of
the tidal interactionto see how often planets spiral into
the star during late stages of stellar evolution and I compared my results 
with recent exoplanet observations. \\\\
\emph{Supervisors: Dr. Alexander J. Mustill \& Dr. Dejan Vinkovi\'{c}}
}
\section{Publications}
\cvitem{2018}{V. Bozza, E. Bachelet, \textbf{F. Bartoli\'{c}}, T. M. Heintz, 
A. R. Hoag, and M. Hun-dertmark.  \emph{VBBINARYLENSING: a public package 
for microlensing light-curve  computation.}, 479:5157–5167,  
October  2018.    doi:   10.1093/mn-ras/sty1791}


\section{Skills}
\cvitem{Programming}{Extensive experience in \texttt{Python}, together 
with libraries \texttt{NumPy}, 
\texttt{Pandas}, \texttt{matplotlib}, and \texttt{Jupyter} notebooks. 
Besides Python. Intermediate experience with \texttt{C/C++}, and have written
code combining Python and C++. I have used \texttt{C\#} in the past.}
\cvitem{Data science}{Extensive experience with the probabilistic 
programming library \texttt{PyMC3} together with \texttt{theano}. 
I have implemented simple 
machine learning models in \texttt{Tensorflow} and \texttt{PyTorch}.}
\cvitem{Statistics}{Extensive experience with Bayesian 
modeling and MCMC methods. Some experience with classification problems.}
\cvitem{Other technical}{\texttt{Git} version control, \texttt{Vim}, \LaTeX, 
\texttt{Linux} systems, \texttt{Bash} shell.}
\cvitem{Communication}{ I have given talks at conferences and meetings in academia, 
as well as talks to the general public. I have given tutorials 
for an undergraduate course in in astronomy. 
I have experience with describing complex statistical methods to a lay audience.}
\cvitem{Organization}{I am organizing a regular meeting on Data Science at my
department and co-organising a weekly meeting on coding.}
\cvitem{Mentoring}{I have supervised a summer student working on a machine learning 
project in astronomy.}

\section{Awards, Competitions and Honors}
\cvitem{2017}{\emph{STFC PhD studentship in data intensive science}, part of 
the Scottish Data-Intensive Science Triangle (ScotDIST).}
\cvitem{2016}{\emph{Erasmus+ Internship } scholarship.}
\cvitem{2015}{\emph{Dean's Award for undergraduate academic excellence}, University of Split.}
\cvitem{2014}{\emph{Erasmus+ Exchange }scholarship.}

\section{Talks}
\cvitem{01/2019}{\emph{Gaussian	process	models of correlated noise in microlensing lightcurves},
 "23rd International Microlensing Meeting", Flatiron Institute, New York, USA.}
\cvitem{12/2018}{\emph{Automatic Differentiation}, talk at the weekly "Code  \& Cake"
meeting, St Andrews, UK.}
\cvitem{08/2018}{\emph{Bayesian modeling of gravitational microlensing events}, 
"ScotDIST Annual Conference", Glasgow, UK.}
\cvitem{05/2018}{\emph{Fitting a model to data}, talk at the weekly "Code \& Cake" 
meeting, St Andrews, UK.}
\cvitem{04/2018}{\emph{Modeling gravitational microlensing events}, 
short talk at the "Scottish Exoplanet/Brown Dwarf Spring meeting", St Andrews, UK.}
\cvitem{09/2017}{\emph{Dynamics of circumbinary planets and the role of mean-motion 
resonances}, astronomy department seminar, Lund, Sweden.}
\cvitem{05/2017}{\emph{Dynamics of circumbinary planets}, talk at 
"Astro Journal Club", Lund, Sweden.}
\cvitem{05/2015}{\emph{The Giants of Ragnar\"ok}, bachelor's thesis public defense, 
Lund, Sweden.}

\section{Outreach activities}
\cvitem{12/2018}{\emph{How to find exoplanets}, talk for the general public at the St 
Andrews Observatory Open Night, organised by the St Andrews Centre for Exoplanet Science 
in St Andrews, UK.}
\cvitem{2017}{\emph{ALVA astronomy club}, Board member. Lund, Sweden.}
\cvitem{09/2014}{\emph{European Researchers' Night}, volunteer, Split, Croatia.}
\cvitem{04/2014}{\emph{Festival of Science}, volunteer, Split, Croatia.}

\section{Supervision}
\cvitem{Summer 2018}{Drew Millard, summer student at St Andrews. I supervised him on a 
project on using machine learning to classify astrophysical time series data.}

\section{Teaching experience}
\cvitem{2019}{Tutor for the course \emph{Astronomy and Astrophysics 2}, an 
undergraduate astronomy course, University of St Andrews.}
\cvitem{2018}{Tutor for the course \emph{Astronomy and Astrophysics 1}, an 
introductory undergraduate astronomy course, University of St Andrews.}
\end{document}

